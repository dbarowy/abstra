\documentclass[10pt]{article}

% Lines beginning with the percent sign are comments
% This file has been commented to help you understand more about LaTeX

% DO NOT EDIT THE LINES BETWEEN THE TWO LONG HORIZONTAL LINES

%---------------------------------------------------------------------------------------------------------

% Packages add extra functionality.
\usepackage{times,graphicx,epstopdf,fancyhdr,amsfonts,amsthm,amsmath,algorithm,algorithmic,xspace,hyperref}
\usepackage[left=1in,top=1in,right=1in,bottom=1in]{geometry}
\usepackage{sect sty}	%For centering section headings
\usepackage{enumerate}	%Allows more labeling options for enumerate environments 
\usepackage{epsfig}
\usepackage[space]{grffile}
\usepackage{booktabs}
\usepackage{forest}
\usepackage{enumitem}   
\usepackage{fancyvrb}

% This will set LaTeX to look for figures in the same directory as the .tex file
\graphicspath{.} % The dot means current directory.

\pagestyle{fancy}

\lhead{Final Project}
\rhead{\today}
\lfoot{CSCI 334: Principles of Programming Languages}
\cfoot{\thepage}
\rfoot{Fall 2023}

% Some commands for changing header and footer format
\renewcommand{\headrulewidth}{0.4pt}
\renewcommand{\headwidth}{\textwidth}
\renewcommand{\footrulewidth}{0.4pt}

% These let you use common environments
\newtheorem{claim}{Claim}
\newtheorem{definition}{Definition}
\newtheorem{theorem}{Theorem}
\newtheorem{lemma}{Lemma}
\newtheorem{observation}{Observation}
\newtheorem{question}{Question}

\setlength{\parindent}{0cm}


%---------------------------------------------------------------------------------------------------------

% DON'T CHANGE ANYTHING ABOVE HERE

% Edit below as instructed

\begin{document}
  
\section*{Language Specifications}

Sammy Sasaki, Mico Mendoza

\subsection{Introduction}

Geometric abstraction is an abstract art form that primarily uses simple geometric shapes painted with flat colors and placed in non-illusionistic planar space. Many people enjoy geometric abstract art, but sometimes they can be hard to create or find, especially for people with little to no artistic skills. 

This is a domain specific programming language that offers users a way to easily create geometric abstract art. It lets users create art that is customized to their own liking, in terms of color scheme, size, shapes, and other parameters.



\subsection{Design Principles}

The main guiding principles in this programming language are simplicity and usability. We want our language to be simple and easy to use, and give the user as much power as possible to create the art that they are envisioning for themselves. We designed our language so that the users have the power to easily customize the primary visual features of a geometric abstract art -- the 2D geometirc shapes and colors/color scheme. Other visual features that we are keeping in mind as we develop our language further are the number of edges for each shape, the relative sizes of the shape, the positions of the shape, the colors of adjacent shapes, the size of the canvas that we are filling up with shapes (and gaps between shapes), etc.  

\subsection{Examples}

\begin{verbatim}
    5 shapes, scheme: greyscale, max 8 edges
\end{verbatim}

\begin{verbatim}
    7 shapes, scheme: rainbow, max 6 edges
\end{verbatim}

\begin{verbatim}
    2 shapes, scheme: red green blue purple, max 3 edges
\end{verbatim}

\subsection{Language Concepts}

Each program takes 3 parameters to specify the geometric abstract art a programmer wants to create. The number of shapes and the number of maximum edges for any given shape are primitive types (int) in this language, while the color scheme is a combined form of the primitive type color (string), where a programmer can choose from preset color schemes or customize their own (must be 4 or more distinct colors).

\subsection{Minimal Formal Syntax}

\begin{Verbatim}[commandchars=\\\{\}]
<expr>   ::= <num> shapes, scheme: <scheme>, max <num> edges
<num>    ::= (is any positive integer)
<scheme> ::= <preset>
          | <color>\textsuperscript{4+}
<color>  ::= red | orange | yellow | green | blue | indigo | purple
<preset> ::= greyscale | rainbow
\end{Verbatim}

\subsection{Minimal Semantics}


\begin{enumerate}[label=(\roman*)]
\item There are 3 primitive types: number of shapes and max edges, which are both ints, and color, which is a string.
\item Our program produces an image filled with connected shapes, each painted with a color based on the color scheme, which is a combined form of 4 or more colors.
\item The types are ints for num shapes and max edges, and algebraic data type of strings for the schemes, which is another ADT that can be a preset or user-inputted. The user-inputted colors are also an ADT of strings for the colors. 

\item 

\begin{center}
    \begin{forest}
      [$\langle expr\rangle$
        [$\langle num\rangle$
          [$5$
          ]
        ]
        [{shapes, }]
        [{scheme: }]
        [$\langle scheme \rangle$
          [$\langle preset\rangle$
            [greyscale]
          ]
        ]
        [{, max}]
        [$\langle num\rangle$
          [$8$
          ]
        ]
        [{ edges}]
      ]
    \end{forest}
  \end{center}

\begin{center}
    \begin{forest}
      [$\langle expr\rangle$
        [$\langle num\rangle$
          [$7$
          ]
        ]
        [{shapes, }]
        [{scheme: }]
        [$\langle scheme \rangle$
          [$\langle preset\rangle$
            [rainbow]
          ]
        ]
        [{, max}]
        [$\langle num\rangle$
          [$6$
          ]
        ]
        [{ edges}]
      ]
    \end{forest}
  \end{center}

\begin{center}
    \begin{forest}
      [$\langle expr\rangle$
        [$\langle num\rangle$
          [$2$
          ]
        ]
        [{shapes, }]
        [{scheme: }]
        [$\langle scheme \rangle$
          [$\langle color\rangle$
            [red]
          ]
          [$\langle color\rangle$
            [green]
          ]
          [$\langle color\rangle$
            [blue]
          ]
          [$\langle color\rangle$
            [purple]
          ]
        ]
        [{, max}]
        [$\langle num\rangle$
          [$3$
          ]
        ]
        [{ edges}]
      ]
    \end{forest}
  \end{center}

  \item The programs don't take any user input outside of the original program lines. The programs output svg files that when opened display a colored canvas containing the art piece. The evaluation would be done using svg. First a string of svg commands would be created when evaluating the tree, and that would be fed into a file which can be opened.



\end{enumerate}

\subsection{Limitations/Remaining Work}
This is a minimally working version of our language, and here are limitations/items that need to be worked on further:
\begin{enumerate}
\item Currently, only the rainbow color scheme works
\item We have to improve how the shapes are placed in the canvas so that no gaps/unfilled spaces remain
\item We have to assign distinct colors for adjacent shapes
\end{enumerate}

% DO NOT DELETE ANYTHING BELOW THIS LINE
\end{document}
